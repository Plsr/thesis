\section{Thema der Abschlussarbeit}
Thema der Abschlussarbeit soll die Umseztung und Gestaltung einer Web-App auf Basis von bereits im Praxisprojekt erarbeiteten Konzepten sein.
Die Web-App soll vorrangig Studenten dabei unterstützen, in ihren Gestaltungen von Artefakten, die im Hochschulalltag benötigt werden, eine Grundqualtität zu sichern.
Ziel der Abschlussarbeit soll es sein, ein funktionierendes und qualitativ hochwertiges Produkt zu erstellen.
Dabei stellen sowohl der Weg vom Konzept und Proof of Concept zum fertigen Produkt als Ablauf, als auch die Umsetzung in den verschiedenen Bereichen die Themengebiete dar, die in der Abschlussarbeit behandelt werden sollen.

Im Praxiprojekt wurde versucht, in verschiedenen Bereichen der visuellen Gerstaltung Regeln und Richtlinien zu definieren, die später von einen Computerprogramm bewertet werden können. Dieser Versuch war in verschiedenen Bereichen unterschiedlich erfolgreich. Um die Konzeptarbeit in der Abschlussarbeit möglichst gering zu halten, soll sich in dieser Hauptsächlich auf die bereit gut abgedeckten Bereiche konzentriert werden. Eine auflistung dieser Bereiche findet sich in \autoref{chapter:setting}: "\nameref{chapter:setting}".
Weiterhin wurde im Bereich Typografie ein Proof of Concept erstellt, auf dem aufgebaut werden kann.

Im Folgenden sollen die sich für die Abschlussarbeit ergebenden Themengebiete kurz erläutert werden.

\subsection{Genereller Aufbau der Anwendung}
\label{chapter:setting}
Im Rahmen des Praxisprojektes wurden bereits grundlegende Konzepte zu Inhalten der Anwendung in verschiedenen Bereichen erstellt. Da hier nicht in allen Bereichen zufriedenstellende Inhlate gefunden werden konnten, sollen diese als erster Schritt der Abschlussarbeit auf ihre Umsetzbarkeit überprüft werden. Weiterhin muss festgestellt werden, welche Gebiete gegbenenfalls noch weiter vertieft werden müssen.

Die wichtigsten Beispiele für noch nicht hinreichend definierte Bereiche stellen hier der Einstiegspunkt und der Endpunkt der Anwendung dar.
Viele der Kernbereiche der Anwendung greifen auf die verschiedenen Limitationen zurück, die mit derm Gestaltungsziel des Nuzters (Plattform, Medium etc.) einher gehen. Diese müssen im ersten Schritt vom Nutzer definiert werden. Hier soll definiert werden, welche Informationen benötigt werden und wie der Nutzer diese der Anwendung mitteilen soll.
Auch muss geklärt werden, in welcher Form die Anwendung Informationen an den Nutzer weiter gibt. So werden dem Nuzter bereits währen der Nutzung Informationen durch seine Interaktion mit der Anwendung mitgeteilt, jedoch besteht bisher kein Konzept, nach dem diese dem Nutzer am Ende der Anwendung noch einmal gebündelt zur weiterverwendung angeboten werden.

In einer ersten funktionierenden Version der Anwednung sollten die folgenden Bereiche enthalten sein:

\begin{itemize}
  \item Einstieg
  \item Typographie
  \item Layout \& Grid
  \item Farben
  \item Outcomes
\end{itemize}

Je nach Zeitlichem verlauf können diese noch um weitere Themengebiete erweitert werden.

\subsection{Technologien und Architektur}
Auch die Architektur der Anwendung und die verwendete Technologie [Technologie hier das richtige Wort?] soll diskutiert werden.

Die Anwendung soll mit Hilfe des Framework React.js umgesetzt werden. React.js ist ein Javascript-Framework, das 2013 von Facebook entwickelt wurde. \cite{gackenheimer2015react} beschreibt React wie folgt:

\begin{quote}
[...] React is a way to describe the user interface of an application and a mechanism to change that over time as data changes. React is made with declarative components that describe an interface. React uses no observable data binding when building an application. React is also easy to manipulate, because you can take the components you create and combine them to make custom components that work as you expect every time because it can scale.
\end{quote}

Der Fokus des Framworks liegt also auf interaktiven User Interfaces, in denen sich die enthaltenen Daten häufig ändern. Da die Anwenundung dem Nuzter ineraktives Feedback zu seinen Eingaben geben soll, eigenet sich React gut für diese Zwecke. \\
Der Prrof of Concept im Praxisprojekt wurde bereits erfolgreich mit Hilfe von React.js umgesetzt und unterstützt die Entscheidung zur Wahl des Frameworks weiter.

Trotzdem sollen in der Abschlussarbeit mögliche andere Frameworks zur umsetzung diskutiert und verglichen werden. Hier bieten sich vor allem andere Javascript Frameworks wie zum Beispiel Ember.js, Angular.js, Vue,js oder Backbone.js an.

Da für die Anwendung keine persistente Datenhaltung benötigt wird, kann auf eine klassische Client-Server-Architektur verzichtet werden und die komplette Logik kann auf dem Client ausgeführt werden.\\
Trotzdem sollte auch im Client eine Architektur entworfen werden. Wie oben bereits erwähnt hängen Teile der Anwendung voneinander ab und frühere Schritte beeinflussen spätere. Deshalb soll zu beginn ein Modell der Anwendung und der Datenhaltung angefertigt werden. Für die Datenhaltung soll auf redux zurückgegriffen werden.

Redux verwaltet den Zustand (oder \textit{state}) einer Anwendung und sorgt dafür, dass dieser Konsistent bleibt. Diese Konsistenz wird durch drei Grunprinzipien von redux erreicht \cite{threeprinciplesredux}:

\begin{enumerate}
  \item Der Zustand wir nur an einer zentralen Stelle verwaltet.
  \item Der Zustand kann nur gelesen und nicht direkt manipuliert werden.
  \item Der Zustand kann nur durch \textit{pure functions}\footnotemark verändert werden.
\end{enumerate}

Redux wurde bereits beim Proof of Concept für den Bereich Typographie im Praxisprojekt verwendet. Hier konnte der Nutzer verschiedene Werte ändern, die sich auf verschiedene DOM-Elemente auswirkten. Redux kontne dafür soregen, dass sich diese Änderungen nicht gegenseitig überschreiben und der Zustand der Anwendung konsistent bleibt.

[ Vielleicht diagramm, wie das im PP gemacht wurde? ]

\subsection{Gestaltung der Anwendung}
Auch die Gestaltung der Anwendung soll Teil der Bachelorarbeit sein. Hierbei soll darauf geachtet werden, dass die Gestaltung der Anwendung möglichst in den Hintergrund tritt, um den Nutzer nicht von der Gestaltung abzulenken, die er mittels der Anwendung erstellt.
Der Designprozess soll hierbei auf das komponentenbasierte Verhalten von React.js angepasst sein. Ganze Seiten sollen  nur in Form von Wireframes erstellt werden, die einzelnen Komponenten auf den Seiten dann im Detail gestaltet werden. Am Ende des Designporzesses soll Verzeichnis von Komponenten entstehen, die in der Anwendung verwendet werden können.

Um dieses Verzeichnis auch in die Anwendung übernehmen zu können, soll auf React-Storybook zurück gegriffen werden. React Storybooks ermöglicht es, einen \textit{living styleguide} für die Verwendung in der Anwendung zu erstellen. Der Code für die verschiedenen Komponenten im Stroybook ist dem, der in der Anwendung verwendet wird dabei sehr ähnlich. Im Stroybook können außerdem verschiedene Stati und Funktionen der Komponenten abgebildet werden. Abb. X zeigt dies am Beispiel von XYZ.

[ Abbildung mit Beispiel von Storybooks ]

Durch die komponentenbasierte Form von React bietet es sich an, CSS auch mit in diesen Komponenten zu verarbeiten, um sowohl die funktion als auch das Aussehen in einer in sich geschlossenen Komponente zu speichern. [ Satz ist große kaka, muss noch besser werden. ]


\section{Relevanz des Themas}
Die Relevanz der Anwendung an sich wurde bereits im Praxisprojekt erläutert, daher soll hier nur eine Kurzfassung der Beweisführung folgen:
[Lindgaard et al] haben gezeigt, dass Menschen sich in nur 50ms ein Urteil über die Gestaltung einer Website bilden. Der erste Eindruck spielt bei Menschen auch bei der späteren Bewertung einer Sache noch eine wichtige Rolle.  Von diesem ersten Eindruck lassen sich Menschen nicht ohne weiteres wieder abbringen.
Daher ist es wichtig, dass auch Artefakte in Modulen, in denen die Gestaltung nicht bewertet wird, ein solides Design aufweise, um den ersten Eindruck so positiv wie möglich zu gestalten.

Hier seien aber noch zwei weitere Quellen aufgeführt, die die korrektheit dieser These [ist das eine These?] weiter stützen.
[Hier noch die zwei Quellen vorstellen]

Neben der Relevanz des Produktes, das in Rahmen der Arbeit entstehen soll ist aber auch das Meta-Thema der Arbeit zu rechtfertigen: Die Entwicklung von einem Konzept oder PoC zu einem fertigen Produkt.
Dieses Thema ist meiner Meinung nach für einen Studenten der Medieninformatik sehr passend, da in diesem Thema viele Aspekte das gesamten Studiums vereint werden. In den weiter oben genannten Bereichen kommen Inhalte aus den Modulen AP 1\&2, GdvK, WBA 1, ST,  PM und EIS vor. Daher bietet das Thema eine gute Verbindung zwischen den verschiedenen Disziplinen innerhalb des Studiums, verbunde mit einer wissenschaftlichen Disukssion verschiedener Vorgehensweisen und Abläufe.

\section{Durchführung der Arbeit}
\subsection{Zielsetzung}
Ziel der Arbeit soll es sein, ein funktionierendes Produkt zu entwickeln, das von Nutzern gewinnbringend verwendet werden kann.
Weiterhin soll die Arbeit Abläufe und Entscheidungen für die Umsetzung eine Projektes, wie sie auch in der Industrie Anwendung finden, im kleinen Rahmen wiederspiegeln.

\subsection{Erste Gliederung}
Hier soll auch eine erste Gliederung der Arbeit erfolgen. Diese Gliederung besitzt keinen Anspruch auf Vollständigkeit und kann während der Durchführung der Arbeit noch verändert werden.
Gerade für den zweiten Teil gestaltet es sich schwierig, konkretere Voraussagen über den Inhalt zu machen, da viele der Artefakte, auch den sich dieser bezieht, noch nicht existieren.

\begin{itemize}
  \item Einleitung
  \item Relevanz des Themas
  \item Zielsetzung der Arbeit
  \item 1.Teil: Theoretische Grundlagen
  \begin{itemize}
    \item Struktur der Anwendung
    \begin{itemize}
      \item Stand Praxisprojekt
      \item Einstieg
      \item Outcome
    \end{itemize}
    \item Verwendete Technologie
    \begin{itemize}
      \item Vue.js
      \item Angular.js
      \item React.js
    \end{itemize}
    \item Vorgehen bei der Gestaltung
  \end{itemize}
  \item 2.Teil: Umsetzung
  \begin{itemize}
    \item Gestaltung
    \item Architektur der Anwendung
    \begin{itemize}
      \item redux
    \end{itemize}
    \item React Storybooks
    \item Redux Store
    \item Gestaltung
    \item Algorithmen
  \end{itemize}
  \item Fazit
  \item Ausblick
\end{itemize}

\subsection{Zeitplan und Organisation}
Die Arbeit soll so bald wie möglich, spätestens aber am 31.5.2017 angemeldet werden. Da kein erhöhter empirischer Anteil besteht, ist die reguläre Dauer von 9 Wochen zur Beendigung der Arbeit vorgesehen. Die Arbeit soll also (je nach Anmeldung) spätestens in der ersten Augustwoche 2017 abgegeben werden.

Als zweitprüferin ist Dipl. Des. Liane Kirschner vorgesehen, die aktuell bei der Railslove GmbH in Köln tätig ist.

\footnotetext{\textit{pure functions} sind funktionen, die keine Nebeneffekte aufweisen, also beim Aufruf mit den gleichen Argumenten zwangsweise die gleichen Ergebnisse liefern \cite{Haverbeke201412}. So kann eine Funktion, deren Ergbnis beispielsweise vom aktuellen Datum abhängt, niemlas \textit{pure} sein.}
