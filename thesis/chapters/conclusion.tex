\chapter{Fazit \& Ausblick}
\thispagestyle{fancy}

Bei der Textgestaltung und automatischen Änderung von Abbildungsnummern, Querverweisen,
Seitenzahlen, Gliederungen, Literaturhinweisen etc. bietet sich der Rückgriff
auf moderne Textverarbeitungsprogramme an. Nutzen Sie diese zur besseren Lesbarkeit
und Strukturierung des Textes, aber vermeiden Sie überflüssige Spielereien. Da
besonders bei Textdokumenten mit eingebundenen Objekten wie Bildern, Formeln

\section{Zielerreichungsgrad}
Zunächst lässt sich feststellen, dass im Rahmen der vorliegenden Arbeit eine Anwendung Konzipiert, Gestaltet und Umgesetzt werden konnte, die nach den in Kapitel 1 definierten Maßstäben eine Marktfähige Anwendung ist.
Die Anwendung biete, sowohl durch ihre Gestaltung, als auch durch ihre Programmierung eine befriedigende Nutzererfahrung. Dem Nutzer gegenüber wird deutlich kommuniziert, an welchem Punkt der Anwendung er sich befindet und er erhält einen Mehrwert aus der Verwendung der Anwendung.
Durch das Hosting und die damit verbunden Domain ist die Anwendung weiterhin für Nutzer leicht zugänglich. Außerdem ist sie auch für Nutzer, die an einer Mitarbeit an der Anwendung interessiert sind durch das öffentliche Repository und die darin erläuterten Wege zur Mitarbeit leicht zugänglich.
Weiterhin ist die Anwendung eine ersten Gruppe von Personen bekannt. Hier fällt es schwer, den Grad der Zielerreichung festzulegen, da es sich zum Einen schwierig gestaltet, die Bekanntheit zu messen, die Bekanntmachung der Anwendung zum Anderen aber auch ein andauernder Prozess ist.

Auch die Mitarbeit an der Anwendung durch die Community lässt sich zu diesem Zeitpunkt noch nicht beurteilen, da die Anwendung etwa Zeitgleich mit Fertigstellung dieser Arbeit veröffentlicht wurde. Es lässt sich jedoch feststellen, dass die Grundsteine für eine mögliche Mitarbeit von anderen gelegt wurden.

An dieser Stelle sei außerdem noch einmal der geplante Umfang der Anwendung angesprochen. Hier wurde der Bereich “Layouts \& Grid” zwar zu beginn als wünschenswert definiert, jedoch konnte dieser in der ersten Version der Anwendung (wie in Kapitel 3.6 ausgeführt) nicht implementiert werden. Dies ist vor dem Gedanken, dass hier ein MVP erstellt werden sollte jedoch hinnehmbar.

\section{Ausblick}
Wie bei der Entwicklung jeder Anwendung ist es auch hier schwer, an einem bestimmten Punk von einer “fertigen Anwendung” zu sprechen. Es konnte eine Anwendung erstellt werden, die gemessen am Rahmen der Arbeit und dem vorgegebenen Arbeitsaufwand als fertig bezeichnet werden kann. Für sich betrachtet, bieten sich noch viele Möglichkeiten, die Anwendung zu erweitern und in ihrem aktuellen Stand zu verbessern.
Einige potentielle Verbesserungen des aktuellen Standes wurden bereits in den zugehörigen Kapiteln angesprochen. Diese wurden als Issues mit in der Github Repository übernommen, um so bereits erste Schritte für die Weiterentwicklung zu gehen.

Weiterhin bieten sich Erweiterungen der Anwendung in From von neuen Themengebieten an. Diese können die in Kapitel 1 definierten sein, jedoch können durchaus auch komplett neue, noch gar nicht bedachte Themengebiete auftreten.
Ein weiterer, interessanter Punkt ist die intensivere Nutzung der Scopes. Diese haben momentan nur sehr subtile Auswirkungen, für jeden Anwendungsfall durchläuft der Nutzer jedoch die gleiche Anwendung, die sich nur in Details unterscheidet. Mit weiteren Anwendungsgebieten kann auch hier eine erhöhte diveristät der verschiedenen Scopes angenommen werden.

Generell lässt sich die Wahl von React.js zur Umsetzung der Anwendung auch weiterhin vertreten. Die Library bietet zum einen die nötigen Möglichkeiten zur Erweiterung der Anwendung, zum anderen aber auch das Potential, auf Dauer als interessantes Projekt zu wirken.


\section{Fazit}
Für mich persönlich war die Arbeit an dem Projekt im Rahmen dieser Arbeit sehr spannend. Hier konnte ich viele Rollen bekleiden, die in einem größeren Projekt wahrscheinlich von verschiedenen Personen bekleidet werden und so die verschiedenen Aufgabenbereiche kennen lernen.
Das Ökosystem um React.js war dabei ein sehr spannendens. Man merkt diesem das Junge alter der Library dabei an. Viele Best Practices ändern sich häufig und sind nicht so sehr verinnerlicht wie in anderen, älteren Libraries wie beispielsweise Ruby on Rails. Häufig kann man dabei mitverfolgen, wie sich best practices im Rahmen von Diskussionen auf Seiten wie Github herausstellen. Weiterhin werden recht häufig neue Versionen von Bibliotheken veröffentlicht, die dafür sorgen, dass das Ökosystem sehr lebendig bleibt.

Neu war für mich persönlich auch die Situation, ein Produkt mit dem Hintergedanken zu entwickeln, dass dieses nicht nur jemand nutzen soll, sondern auch jemand komplett fremdes Einblick in den Code bekommen und an diesem mitarbeiten könnte.

Insgesamt war die Arbeit an diesem Projekt eine sehr spannende, die sicherlich auch außerhalb des Rahmens dieser Arbeit noch weiter gehen wird.

[ HIER VIELLEICHT NOCH COOLES ZITAT ]
