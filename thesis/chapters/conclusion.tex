\chapter{Fazit}
\thispagestyle{fancy}

In der vorliegenden Arbeit konnte die Umsetzung einer marktfähigen Anwendung auf Basis von bereits bestehenden Konzepten und einem Proof of Concept gezeigt werden.
Dabei konnten die verschiedene theoretischen und praktischen Schritte, die für eine solche Umsetzung nötig sind, genauer betrachtet werden.
Es konnte ein Einblick in das relativ junge Feld der JavaScript Single Page Applications und die Entwicklung mit diesen gegeben werden. Auf die Gestaltung einer Anwendung, die ihrerseits wiederum zur Gestaltung dient, konnte eingegangen werden, wie auch auf die nötigen Schritte, um eine Anwendung zur Nutzung und zur Weiterentwicklung verfügbar zu machen.

Der Zielerreichungsgrad für die Umsetzung der Anwendung lässt sich gut an den in Kapitel \ref{chap:goal} definierten Zielsetzungen feststellen:

Die Anwendung bietet, sowohl durch ihre Gestaltung und Programmierung, als auch durch ihren Aufbau eine befriedigende Nutzererfahrung. Dem Nutzer ist es möglich, der Anwendung sein Zielmedium mitzuteilen und die Anwendung vermittelt auf dieses Medium optimierte Inhalte. Weiterhin erhält der Nutzer durch die Möglichkeit, seine Ergebnisse als PDF-Datei zu speichern, einen langfristigen Mehrwert aus der Benutzung der Anwendung.

Durch die Auslieferung der Seite durch einen Server und die Verwendung einer prägnanten URL ist die Anwendung für ihre Nutzer einfach zugänglich. Dies gilt durch die Veröffentlichung des Quelltextes der Anwendung weiterhin für Personen, die an der Weiterentwicklung der Anwendung interessiert sind.

Im Bezug auf die Bekanntheit der Anwendung gestaltet es sich schwierig, den Grad der Zielerreichung festzustellen, da die Anwendung etwa Zeitgleich mit Fertigstellung dieser Arbeit veröffentlicht wurde.
Es lässt sich jedoch festhalten, dass die Grundlagen, sowohl für eine Bekanntmachung, als auch für eine Mitarbeit der Community an der Anwendung geschaffen wurden.

An dieser Stelle sei außerdem noch einmal der geplante Umfang der Anwendung angesprochen, der nicht komplett umgesetzt werden konnte.
Wie in Kapitel \ref{chap:grid_dev} angesprochen wurde hier auf die Implementierung eines Teils der Anwendung verzichtet, der maximal einen prototypischen Stand hat, um die Qualität der Anwendung als ganzes auf einem hohen Niveau zu halten.

Wie bei der Entwicklung jeder Anwendung ist es auch hier schwer, an einem bestimmten Punk von einer \textit{fertigen Anwendung} zu sprechen. Es konnte eine Anwendung erstellt werden, die gemessen am Rahmen der Arbeit und dem vorgegebenen Arbeitsaufwand als fertig bezeichnet werden kann, jedoch bieten sich noch einige Möglichkeiten zur Weiterentwicklung.

In den vergangenen Kapiteln wurden bereits Möglichkeiten angesprochen, mit denen der aktuelle Stand der Anwendung weiter verbessert werden kann (beispielsweise durch eine Zielgerichtetere Aufbereitung der PDF-Datei).
Es bieten sich aber auch Erweiterungen in völlig neuen Themengebieten an. Dabei kann es sich sowohl um bereits konzipierte Themengebiete (siehe Kapitel \ref{chap:pp}) handeln, als auch um solche, die bisher nicht definiert wurden. Mit einer zunehmenden Zahl von Nutzern ist hier auch mit einer zunehmenden Zahl von neuen Anforderung zu rechnen.

Abschließend lässt sich feststellen, dass das Ziel, eine marktfähige Webanwendung zu entwickeln, durchaus als erreicht angesehen werden kann. Konzepte, die theoretisch eine Nützlichkeit aufweisen, konnten damit in eine Anwendung überführt werden, die einen realen praktischen Nutzen aufweist.
