\chapter{Entwicklung der Anwendung}
\thispagestyle{fancy}
Bei der Textgestaltung und automatischen Änderung von Abbildungsnummern, Querverweisen,
Seitenzahlen, Gliederungen, Literaturhinweisen etc. bietet sich der Rückgriff
auf moderne Textverarbeitungsprogramme an. Nutzen Sie diese zur besseren Lesbarkeit
und Strukturierung des Textes, aber vermeiden Sie überflüssige Spielereien. Da
besonders bei Textdokumenten mit eingebundenen Objekten wie Bildern, Formeln

\section{Gestaltung}
Bei der Textgestaltung und automatischen Änderung von Abbildungsnummern, Querverweisen,
Seitenzahlen, Gliederungen, Literaturhinweisen etc. bietet sich der Rückgriff
auf moderne Textverarbeitungsprogramme an. Nutzen Sie diese zur besseren Lesbarkeit
und Strukturierung des Textes, aber vermeiden Sie überflüssige Spielereien. Da
besonders bei Textdokumenten mit eingebundenen Objekten wie Bildern, Formeln

\subsection{Wireframes \& Struktur}
Bei der Textgestaltung und automatischen Änderung von Abbildungsnummern, Querverweisen,
Seitenzahlen, Gliederungen, Literaturhinweisen etc. bietet sich der Rückgriff
auf moderne Textverarbeitungsprogramme an. Nutzen Sie diese zur besseren Lesbarkeit
und Strukturierung des Textes, aber vermeiden Sie überflüssige Spielereien. Da
besonders bei Textdokumenten mit eingebundenen Objekten wie Bildern, Formeln

\subsection{High Fidelity Mockups}
Bei der Textgestaltung und automatischen Änderung von Abbildungsnummern, Querverweisen,
Seitenzahlen, Gliederungen, Literaturhinweisen etc. bietet sich der Rückgriff
auf moderne Textverarbeitungsprogramme an. Nutzen Sie diese zur besseren Lesbarkeit
und Strukturierung des Textes, aber vermeiden Sie überflüssige Spielereien. Da
besonders bei Textdokumenten mit eingebundenen Objekten wie Bildern, Formeln

\section{Besonderhieten im Technologie-Stack}
Bei der Textgestaltung und automatischen Änderung von Abbildungsnummern, Querverweisen,
Seitenzahlen, Gliederungen, Literaturhinweisen etc. bietet sich der Rückgriff
auf moderne Textverarbeitungsprogramme an. Nutzen Sie diese zur besseren Lesbarkeit
und Strukturierung des Textes, aber vermeiden Sie überflüssige Spielereien. Da
besonders bei Textdokumenten mit eingebundenen Objekten wie Bildern, Formeln

\subsection{Redux}
Bei der Textgestaltung und automatischen Änderung von Abbildungsnummern, Querverweisen,
Seitenzahlen, Gliederungen, Literaturhinweisen etc. bietet sich der Rückgriff
auf moderne Textverarbeitungsprogramme an. Nutzen Sie diese zur besseren Lesbarkeit
und Strukturierung des Textes, aber vermeiden Sie überflüssige Spielereien. Da
besonders bei Textdokumenten mit eingebundenen Objekten wie Bildern, Formeln

\subsection{React Storybooks}
Bei der Textgestaltung und automatischen Änderung von Abbildungsnummern, Querverweisen,
Seitenzahlen, Gliederungen, Literaturhinweisen etc. bietet sich der Rückgriff
auf moderne Textverarbeitungsprogramme an. Nutzen Sie diese zur besseren Lesbarkeit
und Strukturierung des Textes, aber vermeiden Sie überflüssige Spielereien. Da
besonders bei Textdokumenten mit eingebundenen Objekten wie Bildern, Formeln

\subsection{CSS-Architektur}
Bei der Textgestaltung und automatischen Änderung von Abbildungsnummern, Querverweisen,
Seitenzahlen, Gliederungen, Literaturhinweisen etc. bietet sich der Rückgriff
auf moderne Textverarbeitungsprogramme an. Nutzen Sie diese zur besseren Lesbarkeit
und Strukturierung des Textes, aber vermeiden Sie überflüssige Spielereien. Da
besonders bei Textdokumenten mit eingebundenen Objekten wie Bildern, Formeln

\section{Algorythmen}
Bei der Textgestaltung und automatischen Änderung von Abbildungsnummern, Querverweisen,
Seitenzahlen, Gliederungen, Literaturhinweisen etc. bietet sich der Rückgriff
auf moderne Textverarbeitungsprogramme an. Nutzen Sie diese zur besseren Lesbarkeit
und Strukturierung des Textes, aber vermeiden Sie überflüssige Spielereien. Da
besonders bei Textdokumenten mit eingebundenen Objekten wie Bildern, Formeln

\section{Tests}
Bei der Textgestaltung und automatischen Änderung von Abbildungsnummern, Querverweisen,
Seitenzahlen, Gliederungen, Literaturhinweisen etc. bietet sich der Rückgriff
auf moderne Textverarbeitungsprogramme an. Nutzen Sie diese zur besseren Lesbarkeit
und Strukturierung des Textes, aber vermeiden Sie überflüssige Spielereien. Da
besonders bei Textdokumenten mit eingebundenen Objekten wie Bildern, Formeln
