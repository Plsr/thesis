\chapter{Entwicklung der Anwendung}
\thispagestyle{fancy}
Bei der Textgestaltung und automatischen Änderung von Abbildungsnummern, Querverweisen,
Seitenzahlen, Gliederungen, Literaturhinweisen etc. bietet sich der Rückgriff
auf moderne Textverarbeitungsprogramme an. Nutzen Sie diese zur besseren Lesbarkeit
und Strukturierung des Textes, aber vermeiden Sie überflüssige Spielereien. Da
besonders bei Textdokumenten mit eingebundenen Objekten wie Bildern, Formeln

\section{Gestaltung}
Bei der Textgestaltung und automatischen Änderung von Abbildungsnummern, Querverweisen,
Seitenzahlen, Gliederungen, Literaturhinweisen etc. bietet sich der Rückgriff
auf moderne Textverarbeitungsprogramme an. Nutzen Sie diese zur besseren Lesbarkeit
und Strukturierung des Textes, aber vermeiden Sie überflüssige Spielereien. Da
besonders bei Textdokumenten mit eingebundenen Objekten wie Bildern, Formeln

\subsection{Wireframes \& Struktur}
Bei der Textgestaltung und automatischen Änderung von Abbildungsnummern, Querverweisen,
Seitenzahlen, Gliederungen, Literaturhinweisen etc. bietet sich der Rückgriff
auf moderne Textverarbeitungsprogramme an. Nutzen Sie diese zur besseren Lesbarkeit
und Strukturierung des Textes, aber vermeiden Sie überflüssige Spielereien. Da
besonders bei Textdokumenten mit eingebundenen Objekten wie Bildern, Formeln

\subsection{High Fidelity Mockups}
Bei der Textgestaltung und automatischen Änderung von Abbildungsnummern, Querverweisen,
Seitenzahlen, Gliederungen, Literaturhinweisen etc. bietet sich der Rückgriff
auf moderne Textverarbeitungsprogramme an. Nutzen Sie diese zur besseren Lesbarkeit
und Strukturierung des Textes, aber vermeiden Sie überflüssige Spielereien. Da
besonders bei Textdokumenten mit eingebundenen Objekten wie Bildern, Formeln

\section{Besonderhieten im Technologie-Stack}
Bei der Textgestaltung und automatischen Änderung von Abbildungsnummern, Querverweisen,
Seitenzahlen, Gliederungen, Literaturhinweisen etc. bietet sich der Rückgriff
auf moderne Textverarbeitungsprogramme an. Nutzen Sie diese zur besseren Lesbarkeit
und Strukturierung des Textes, aber vermeiden Sie überflüssige Spielereien. Da
besonders bei Textdokumenten mit eingebundenen Objekten wie Bildern, Formeln

\subsection{Redux}
Bei der Textgestaltung und automatischen Änderung von Abbildungsnummern, Querverweisen,
Seitenzahlen, Gliederungen, Literaturhinweisen etc. bietet sich der Rückgriff
auf moderne Textverarbeitungsprogramme an. Nutzen Sie diese zur besseren Lesbarkeit
und Strukturierung des Textes, aber vermeiden Sie überflüssige Spielereien. Da
besonders bei Textdokumenten mit eingebundenen Objekten wie Bildern, Formeln

\subsection{React Storybooks}
Bei der Textgestaltung und automatischen Änderung von Abbildungsnummern, Querverweisen,
Seitenzahlen, Gliederungen, Literaturhinweisen etc. bietet sich der Rückgriff
auf moderne Textverarbeitungsprogramme an. Nutzen Sie diese zur besseren Lesbarkeit
und Strukturierung des Textes, aber vermeiden Sie überflüssige Spielereien. Da
besonders bei Textdokumenten mit eingebundenen Objekten wie Bildern, Formeln

\subsection{CSS-Architektur}
Für den Umgang mit CSS in React.js bieten sich verschiedene Möglichkeiten an. Nativ sind zwei Vorgehensweisen möglich: Anwenden von CSS über ausgelagerte Stylesheets (wie es in der Regel bei allen Webseiten gemacht wird) oder das verwenden von Inline-Styles.

Das verwenden von klassischen Stylesheets unterscheidet sich nicht sehr von der Verwendung bei einer statischen Webseite. Auch in React werden Klassennamen oder IDs festgelegt, über die dann im Stylesheet das Aussehen definiert wird (natürlich sind auch alle anderen validen CSS Selektoren anwendbar, Klassen und IDs seine hier nur als die populärsten Varianten beispielhaft genannt).
Auch die Verwendung von CSS-Präsporzessoren wie zum Beispiel SCSS stellt kein Problem dar, die kompilierung dieser Dateien muss lediglich in den Build-Prozess von React.js mit eingebunden werden. Eine simplifizierte Anwendung sähe beispielsweise wie folgt aus:

\begin{lstlisting}
<ReactComponent className='myClass'>
 Content
</ReactComponent>
\end{lstlisting}

Die kompilierte Dom-Node wäre dabei

\begin{lstlisting}
  <ReactComponent className='myClass'>
   Content
  </ReactComponent>
\end{lstlisting}

Und könnte im Stylesheet über

\begin{lstlisting}
  .myClass {
    color: red
  }
\end{lstlisting}

Angesprochen werden. Auch die Verwendung von Methodiken wie BEM oder SMACCS ist mit diesem Ansatz ohne Probleme möglich.
Für den Rahmen dieses Projektes wurde sich jedoch gegen diese Vorgehensweise entschieden, da bewusst ein Fokus auf eine komponentenbasierte Architektur gelegt werden sollte. Auch mit einer komponentenbasierten CSS-Architektur ist eine Komponente immer noch auf zwei Orte aufgeteilt: Die Funktion und das Markup, und das Styling.

[HIER LIESSE SICH AUCH DIE DISKUSSION NOCH AUSFÜHREN, DASS EIGENTLICH LEUTE LANGE DAFÜR GESPROCHEN HABEN, AUSSEHEN UND MARKUP ZU TRENNEN UND WARUM DAS HIER OKAY IST]

Um das Styling und die Funktion einer Komponente an einem Ort zu halten, bieten sich inline-styles an. Wie auch bei statischen Webseiten werden inline-styles direkt im \verb|style|-Attribut eines Elementes definiert. In React.js werden diese als JavaScript-Objekt übergeben. Eine Beispielhafte Anwendung findet sich in Quellcode XYZ

\begin{lstlisting}
  const styles = {
    backgroundColor: 'red',
    fontSize: '12px'
  }

  export default function myComponent(props) {
    return (
      <div styles=({styles}) >
  		{props.children}
  	  </div>
    )
  }
\end{lstlisting}

Auf den ersten Blick wirkt die Verwendung von inline-styles problematisch, vielleicht weil diese auf statischen Seiten viele Nachteile mit sich bringen. In einem Komponentenbasierten System wie React.js sind diese Nachteile jedoch nicht present. Durch die Komponentenbasierte Struktur und das damit einhergehende Ziel der Wiederverwendung von Komponenten, müssen Stylingänderungen auch hier nur an einer Stelle vorgenommen werden. Da jede Komponente nur für ihr eigenes Styling verantwortlich ist und nicht für das von Kindern, und auch kein CSS außerhalb der inline-styles verwendet wird (abgesehen von CSS-Reset und einigen globalen Regeln wie der Schriftart), kann es zu keinen Problemen mit der Spezifität von CSS-Regeln kommen.

Allerdings weisen inline-styles einige Limitierungen auf. So können beispielsweise keine Pseudo-Elemente wie \verb|:after| oder \verb|:before| verwendet werden. Auch das definieren von hover-states via \verb|:hover| wird nicht unterstützt.

[HIER NOCH VERWEIS AUF SPEZIFIKATION]

Zwar sind diese Limitierung zum aktuellen Stand des Projektes noch nicht von allzu großer Bedeutung, mit Blick auf eine Weiterentwicklung der Anwendung nach der Abschlussarbeit können diese in Zukunft jedoch eine größere Rolle spielen.
Um diese  Limitierungen zu umgehen, wurde das Framework Aphrodite verwendet. Das Framework erlaubt das Festlegen von styles innerhalb der Komponente ähnlich wie inline-styles, erzeugt aber für jedes neue style-objekt eine einzigartige CSS-Klasse, die via \verb|className| angewendet wird. Die Einzigartigkeit der Klasse wird durch das hinzufügen eines Hauses am Ende des Klassennamens gewährleistet.  Listing XYZ zeigt ein Beispiel

\begin{lstlisting}
  import React from 'react'
  import { StyleSheet, css } from 'aphrodite'

  function myComponent(props) {
  	<div className={css(styles.componentStyles)} >
  		{props.children}
  	</div>

  	const styles = StyleSheet.create({
  		componentStyles: {
  			color: 'blue'
  		}
  	})
  }
\end{lstlisting}

Die Struktur der style-objekte ist dabei der von inline-style-objekten gleich.

Mit der Verwendung von Aphrodite können also Aussehen und Funktion von Komponenten in der gleichen Datei gehalten werden, ohne dabei den Limitierungen von inline-styles zu unterliegen.

\section{Einstieg}
Bei der Textgestaltung und automatischen Änderung von Abbildungsnummern, Querverweisen,
Seitenzahlen, Gliederungen, Literaturhinweisen etc. bietet sich der Rückgriff
auf moderne Textverarbeitungsprogramme an. Nutzen Sie diese zur besseren Lesbarkeit
und Strukturierung des Textes, aber vermeiden Sie überflüssige Spielereien. Da
besonders bei Textdokumenten mit eingebundenen Objekten wie Bildern, Formeln

\section{Typographie}
Bei der Textgestaltung und automatischen Änderung von Abbildungsnummern, Querverweisen,
Seitenzahlen, Gliederungen, Literaturhinweisen etc. bietet sich der Rückgriff
auf moderne Textverarbeitungsprogramme an. Nutzen Sie diese zur besseren Lesbarkeit
und Strukturierung des Textes, aber vermeiden Sie überflüssige Spielereien. Da
besonders bei Textdokumenten mit eingebundenen Objekten wie Bildern, Formeln

\section{Farben}
Bei der Textgestaltung und automatischen Änderung von Abbildungsnummern, Querverweisen,
Seitenzahlen, Gliederungen, Literaturhinweisen etc. bietet sich der Rückgriff
auf moderne Textverarbeitungsprogramme an. Nutzen Sie diese zur besseren Lesbarkeit
und Strukturierung des Textes, aber vermeiden Sie überflüssige Spielereien. Da
besonders bei Textdokumenten mit eingebundenen Objekten wie Bildern, Formeln

\section{Offboarding}
Bei der Textgestaltung und automatischen Änderung von Abbildungsnummern, Querverweisen,
Seitenzahlen, Gliederungen, Literaturhinweisen etc. bietet sich der Rückgriff
auf moderne Textverarbeitungsprogramme an. Nutzen Sie diese zur besseren Lesbarkeit
und Strukturierung des Textes, aber vermeiden Sie überflüssige Spielereien. Da
besonders bei Textdokumenten mit eingebundenen Objekten wie Bildern, Formeln

\section{Algorythmen}
Bei der Textgestaltung und automatischen Änderung von Abbildungsnummern, Querverweisen,
Seitenzahlen, Gliederungen, Literaturhinweisen etc. bietet sich der Rückgriff
auf moderne Textverarbeitungsprogramme an. Nutzen Sie diese zur besseren Lesbarkeit
und Strukturierung des Textes, aber vermeiden Sie überflüssige Spielereien. Da
besonders bei Textdokumenten mit eingebundenen Objekten wie Bildern, Formeln

\section{Tests}
Bei der Textgestaltung und automatischen Änderung von Abbildungsnummern, Querverweisen,
Seitenzahlen, Gliederungen, Literaturhinweisen etc. bietet sich der Rückgriff
auf moderne Textverarbeitungsprogramme an. Nutzen Sie diese zur besseren Lesbarkeit
und Strukturierung des Textes, aber vermeiden Sie überflüssige Spielereien. Da
besonders bei Textdokumenten mit eingebundenen Objekten wie Bildern, Formeln
