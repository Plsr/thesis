\chapter{Einleitung}
\thispagestyle{fancy}
Die nachfolgende Arbeit beschäftigt sich mit der Entwicklung der Webanwendung \textit{25Knots} zu einem Markfähigen Produkt, basierend auf bereits im Praxisprojekt erarbeiteten Konzepten und einem Proof of Concept.\\
Ziel ist dabei nicht nur, die Verwendung durch die Community\footnotemark{} nach Abschluss dieser Arbeit möglich zu machen, sondern diese auch aktiv an der Weiterentwicklung der Anwendung teilhaben zu lassen.\\

Die Anwendung \textit{25Knots} zielt darauf ab, die gestalterische Qualität von Artefakten im Hochschulkontext zu verbessern. Dieses Ziel soll durch die Unterstützung von Studierenden während des Entwicklungsprozesses von Artefakten erreicht werden. Die Anwendung soll den Studierenden dabei eine Möglichkeit bieten, interaktiv Ergebnisse zu produzieren, die für ihren aktuellen Entwicklungsprozess hilfreich sind. Wissen soll dabei eher implizit, durch ein \textit{Learning by Doing} Konzept vermittelt werden.\\

Generell sollen in dieser Arbeit alle Aspekte behandelt werden, die auf dem Weg von einem Prototypen zu einem marktfähigen Produkt eine Rolle spielen. Diese umfassen zum Beispiel die Struktur und Gestaltung der Anwendung, technische Entscheidungen die bei der Umsetzung getroffen werden müssen, die generelle Entwicklung, aber auch Bereiche wie Hosting oder Möglichkeiten zur Weiterentwicklung.
Im Folgenden Kapitel sollen zunächst einige grundlegende Ziele und das generelle Vorgehen bei der Umsetzung erläutert werden

\footnotetext{Als Teil der \textit{Community} wird hier jede Person gesehen, die ein Interessen an der Anwendung besitzt.}

\section{Motivation}
Wie bereits erwähnt basiert das Thema dieser Arbeit auf Konzepten, die im Rahmen des Praxisprojektes  erstellt wurden. Bei der Erstellung dieser Konzepte wurde explizit auf eine spätere Umsetzbarkeit geachtet, so wurden zum Beispiel schon einige grobe Berechnungsmethoden entworfen, die in der Anwendung verwendet werden können.\\
Die Umsetzung des Proof of Concept im Bereich Typographie hat außerdem gezeigt, dass die erstellten Konzepte in einer Anwendung durchaus funktionieren.\\
Weiterhin konnte ich in persönlichen Gesprächen mit verschiedenen Personen innerhalb und auch außerhalb des Hochschulkontextes feststellen, dass eine Umsetzung der erarbeiteten Konzepte durchaus eine reale Zielgruppe besitzt.\\

Die Motivation für diese Arbeit ergibt sich daher aus dem Willen, aus diesem erarbeiteten Konzept einen Mehrwert schaffen zu wollen, der über den Erhalt von \textit{Credit Points} hinaus geht. Dieser Mehrwert entsteht dabei zum einen für mich persönlich, zum anderen aber auch für andere Personen, die einen Mehrwert aus der Existenz einer solchen Anwendung ziehen können.\\
Die generelle Motivation für die Existenz einer Anwendung wie \textit{25knots} lässt sich dem Kapitel \textit{\nameref{sec:relevance}} auf Seite \pageref{sec:relevance} entnehmen.

\section{Zielsetzung}
Ziel der vorliegenden Arbeit soll es sein, ein von der Community verwendbares und erweiterbares Produkt zu entwickeln. Zunächst sei hier ein \textit{verwendbares Produkt} im Rahmen dieser Arbeit definiert.\\
Ein verwendbares Produkt:

\begin{enumerate}
  \item bietet eine befriedigende Nutzererfahrung
  \item ist einfach zugänglich
  \item ist bei potentiellen Nutzern als Hilfsmittel zum Erreichen eines Zieles bekannt
\end{enumerate}

Eine befriedigende Nutzererfahrung bedeutet für die Anwendung konkret, dass diese gut Strukturiert, ansprechend Gestaltet und ohne Probleme verwendbar sein muss. \textit{Ohne Probleme verwendbar} impliziert hierbei eine hohe Qualität des geschriebenen Codes, um Fehler zu vermeiden. Der Großteil der Arbeit wird sich mit der Umsetzung dieses Bereiches beschäftigen.\\
Um die Anwendung einfach zugänglich zu machen, bietet sich das Hosting auf einem Server an (im Gegensatz zu beispielsweise nur der Bereitstellung das Anwendungscodes). Weitere details zu diesem Thema werden im Kapitel \textit{\nameref{chap:hosting}} auf Seite \pageref{chap:hosting} erläutert.\\
Damit die Anwendung genutzt werden kann, müssen ihre potentiellen Nutzer von ihrer Existenz wissen. Mit Blick auf die Zielgruppe bietet sich zunächst eine Bewerbung direkt an der Hochschule, beispielsweise durch die Teilnahme am \textit{Medieninfromatik Showcase} an. Aber auch eine Bewerbung im größeren Rahmen, beispielsweise durch einen Talk beim \textit{DevHouse Friday Köln} ist denkbar.\\

Zuletzt sei auch die erweiterbarkeit der Anwendung konkret definiert. Eine mögliche Erweiterbarkeit bedeutet zunächst, dass der Code der Anwendung öffentlich verfügbar sein muss, beispielsweise auf der Plattform Github\footnotemark{}. Weiterhin muss der Code gut dokumentiert und verständlich geschrieben sein, um einen Einstieg in das bestehende Projekt für Dritte so einfach wie möglich zu halten. Eine erweiterbarkeit bezieht sich aber nicht nur auf geschriebenen Code, sondern kann (und soll) auch auf konzeptioneller Ebene erfolgen. Auch hier muss die Möglichkeit zur Beteiligung so simpel wie möglich gehalten werden. Eine ausführlichere Diskussion findet sich im Kapitel \textit{Release}\\

\footnotetext{\url{https://github.com}}

Aus diesen Punkten kann eine zentrale Forschungsfrage für die Arbeit formuliert werden:\\
\textit{\textbf{Wie kann der Community ein nutzbares und erweiterbares Produkt auf Basis eines Konzeptes und Prototypen bereitgestellt werden?}}\\
Diese lässt sich in zwei Unterfragen unterteilen, die es in dieser Arbeit zu beantworten gilt:
\begin{enumerate}
  \item Durch welche Maßnahmen kann eine aktive Nutzung und Weiterentwicklung durch die Community gewährleistet werden?
  \item Welche technischen Entscheidungen müssen während der Entwicklung getroffen werden?
\end{enumerate}

Auch wenn es Teil der Zielsetzung ist, ein marktfähiges Produkt zu erstellen kann hier nicht davon ausgegangen werden, dass das Endergebnis der Arbeit ein Produkt ist, das als fertig angesehen werden kann. Auch mit Blick auf die spätere Weiterentwicklung durch die Community muss es viel mehr das Ziel sein, eine hochwertige erste Version des Produktes, die als Grundlage für weitere Features dient, also ein \textit{Minimum Viable Product} zu erstellen.


\section{Relevanz des Themas}
\label{sec:relevance}
Die Relevanz der Anwendung an sich wurde bereits im Praxisprojekt erläutert, daher soll hier nur eine Kurzfassung der Erläuterung folgen. Der Grundgedanke der Relevanz ist dabei folgender:
\begin{quote}
  Menschen bilden sehr schnell ein Urteil über die Gestaltung eines Artefaktes und dieses lässt sich nur schwer wieder ändern. Weiterhin wird dieser schlechte erste Eindruck auf andere Bereiche des Artefaktes übertragen und wirkt sich somit unter Umständen auch auf die Gesamtbewertung eines Artefaktes aus.
\end{quote}

Gestützt wird dieser Gedanke durch Studien von \cite{lindgaard2006attention}, \cite{campbell1996fitting}, und \cite{nickerson1998confirmation}.

Unterstützend seien hier noch zwei weitere Quellen aufgeführt, die die Relevanz weiter unterstreichen:
\cite{tractinsky2006evaluating} zeigen, dass die Ergebnisse der Studie von Lindgaard et al. auch mit anderen Parametern bestand haben und unterstreichen weiterhin die Wichtigkeit von guter Gestaltung für eine gute Nutzererfahrung \cite{tractinsky2000beautiful}.

Neben der Relevanz des Produktes, das in Rahmen der Arbeit entstehen soll sei aber auch das übergreifende Thema der Arbeit angesprochen: Die Entwicklung von einem Konzept zu einem fertigen Produkt.
Als abschließende Arbeit für den Studiengang Medieninformatik ist dieses ein passendes Thema, da hier viele Aspekte aus verschiedenen Modulen des gesamten Studiums vereint werden. Daher bietet das Thema eine gute Verbindung zwischen den verschiedenen Disziplinen innerhalb des Studiums und einer wissenschaftlichen Diskussion verschiedener Vorgehensweisen und Abläufe.


\section{Zielgruppe}
Während der Konzeption im Praxisprojekt wurden Studenten der Technischen Hochschule Köln im Studiengang Medieninformatik als Zielgruppe festgelegt. Es wurde aber bereits dort deutlich, dass diese Zielgruppe leicht erweiterbar ist. Somit kann jede Person einen Mehrwert aus diesem Tool ziehen, die ein Artefakt erstellen muss dessen Hauptaugenmerk eigentlich nicht auf der Gestaltung, sondern auf einer bestimmten Funktion liegt. Die fehlende Aufmerksamkeit für die Gestaltung kann im Studium an einer fehlenden Bewertung dieser oder in der Wirtschaft an einem mangelndem Budget liegen. Es entstehen also häufig Situationen, in denen eine solide Gestaltung nicht als wichtig erachtet wird, diese jedoch, wie im vorhergehenden Kapitel erwähnt, durchaus Vorteile mit sich bringt.

Für diese Arbeit werden aber zunächst weiterhin die Studenten des Studienganges Medieninformatik als Zielgruppe definiert, um eine Disparität zwischen dem Konzept und der Umsetzung auszuschließen.

\section{Struktur der vorliegenden Arbeit}
Außerhalb dieser Einleitung teilt sich die Arbeit in drei weitere Kapitel.\\
Im zweiten Kapitel werden zunächst einige Theoretische Grundlagen und Entscheidungen auf einer taktischen Ebene erläutert. Dort findet sich neben einigen strukturellen Erläuterungen auch eine Diskussion verschiedener Technologien, mit denen eine Umsetzung möglich gewesen wäre und ein Einstieg in verschiedene Aspekte der Programmierung mit React.js.\\
Im dritten Kapitel werden einige Aspekte der Umsetzung und Gestaltung der Anwendung angesprochen. Es finden sich beispielsweise Erläuterungen zu den einzelnen Bereichen der Anwendung, aber auch Bereichsübergreifende praktische Themen wie die CSS-Architektur.\\
Das vierte Kapitel beschäftigt sich mit Fragen die im Bezug mit der Veröffentlichung der Anwendung stehen.

\textbf{NOTIZ: Hier wird am Ende noch einmal drüber geschaut, wenn die tatsächliche Struktur etwas deutlicher ist.}
