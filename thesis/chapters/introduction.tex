\chapter{Einleitung}
\thispagestyle{fancy}
Diese Arbeit beschäftigt sich mit der Umsetzung der Webanwendung \textit{25knots}, die eine Verbesserung der gestalterischen Qualität von Artefakten im Hochschulkontext sichern soll.
Das Konzept für die Anwendung wurde vorhergehend bereits im Praxisprojekt erarbeitet und beispielhaft in Form eines Proof of Concept umgesetzt. Im Rahmen der Abschlussarbeit soll nun, aufbauend auf den erarbeiteten Konzepten und dem Proof of Concept, ein marktfähiges Produkt erstellt werden. Dieses Produkt soll weiterhin von der Community\footnotemark nicht nur verwendet, sondern auch aktiv weiterentwickelt werden können.\\

Generell sollen in dieser Arbeit alle Aspekte behandelt werden, die auf dem Weg von einem Prototypen zu einem marktfähigen Produkt eine Rolle spielen. Diese umfassen zum Beispiel die Struktur und Gestaltung der Anwendung, technische Entscheidungen die bei der Umsetzung getroffen werden müssen, aber auch Bereiche wie Hosting oder Möglichkeiten zur Weiterentwicklung.
Im Folgenden sollen zunächst einige grundlegende Ziele und das generelle Vorgehen bei der Umsetzung definiert werden.

\footnotetext{Als Teil der \textit{Community} wird hier jede Person gesehen, die ein Interessen an der Anwendung besitzt.}

\section{Motivation}
Wie bereits erwähnt wurde das Konzept für die Anwendung bereits im Praxisprojekt erstellt. Dabei wurde viel Zeit und Energie investiert um sicher zu stellen, dass diese Konzepte auch Umsetzbar sind. So wurden zum Beispiel bereits viele (grobe) Algorithmen entworfen, die in der Anwendung verwendet werden können.
Des weiteren konnte in persönlichen Gesprächen mit verscheidenen Personen innerhalb und auch außerhalb des Hochscholkontextes festgestellt werden dass eine Umsetzung der erarbeiteten Konzepte durchaus eine Relevanz besitzt und von Personen verwendet werden würde.\\

Die Motivation für diese Arbeit ergibt sich daher aus dem Willen, aus diesem erarbeiteten Konzept einen Mehrwert schaffen zu wollen, der über das verdienen von Credit Points hinaus geht. Die generelle Motivation für eine Anwendung wie \textit{25knots} lässt sich Kapitel \textit{\ref{sec:relevance}: \nameref{sec:relevance}} entnehmen

\section{Zielsetzung}
Ziel der vorliegenden Arbeit soll es sein, ein von der Community verwendbares und erweiterbares Produkt zu erstellen. Zunächst sei hier ein \textit{verwendbares Produkt} im Rahmen dieser Arbeit definiert. Ein verwendbares Produkt:

\begin{enumerate}
  \item bietet eine befriedigende Nutzererfahrung
  \item ist öffentlich zugänglich
  \item ist bei potentiellen Nutzern als Hilfsmittel zum Erreichen eines Zieles bekannt
\end{enumerate}

Eine befriedigende Nutzererfahrung bedeutet für die Anwendung konkret, dass diese gut Strukturiert, ansprechend Gestaltet und ohne Probleme verwendbar sein muss. \textit{Ohne Probleme verwendbar} impliziert hierbei eine hohe Qualität des geschriebenen Codes, um Fehler zu vermeiden. Der Großteil der Anwendung wird sich mit den Umsetzungen dieser Bereiche beschäftigen.

Der einfachste Weg, um eine Webanwendung öffentlich zugänglich zu machen ist in der Regel, diese auf einem Server zu hosten. Weiter details dieser Entscheidung werden im Kapitel \textit{Hosting} erläutert.

Damit diese genutzt werden kann, müssen potentielle Nutzer der Anwendung von dessen Existenz wissen. Mit Blick auf die Zielgruppe bietet sich zunächst eine Bewerbung direkt an der Hochschule, beispielsweise durch die Teilnahme am \textit{Medieninfromatik Showcase} an. Aber auch eine Bewerbung im größeren Rahmen, beispielsweise durch einen Talk beim \textit{Webmontag Köln} ist denkbar.\\

Zuletzt sei auch die erweiterbarkeit der Anwendung konkret definiert. Eine mögliche Erweiterbarkeit bedeutet zunächst, dass der Code der Anwendung öffentlich verfügbar sein muss, beispielsweise auf der Plattform \textit{Github}. Weiterhin muss der Code gut dokumentiert und verständlich geschrieben sein, um einen Einstieg in das bestehende Projekt zu einfach wie möglich zu halten. Eine erweiterbarkeit bezieht sich aber nicht nur auf geschriebenen Code, sondern kann auch auf konzeptioneller Ebene erfolgen. Auch hier muss die Möglichkeit zur Beteiligung so simpel wie möglich gehalten werden. Eine ausführlichere Diskussion findet sich im Kapitel \textit{Release}\\

Aus diesen Punkten kann eine zentrale Forschungsfrage für die Arbeit formuliert werden:\\
\textit{Wie kann der Community ein nutzbares und erweiterbares Produkt auf Basis eines Konzeptes und Prototypen bereitgestellt werden?}

Diese lässt sich in zwei Unterfragen unterteilen, die es zu beantworten gilt:

\begin{enumerate}
  \item Durch welche Maßnahmen kann eine aktive Nutzung und Weiterentwicklung durch die Community gewährleistet werden?
  \item Welche technischen Entscheidungen müssen während der Entwicklung getroffen werden?
\end{enumerate}

\section{Relevanz des Themas}
\label{sec:relevance}
Die Relevanz der Anwendung an sich wurde bereits im Praxisprojekt erläutert, daher soll hier nur eine Kurzfassung der Erläuterung folgen. Der Grundgedanke der Relevanz ist dabei folgender:
\begin{quote}
  Menschen bilden sehr schnell ein Urteil über die Gestaltung eines Artefaktes und dieses lässt sich nur schwer wieder ändern. Weiterhin wird dieser schlechte erste Eindruck auf andere Bereiche des Artefaktes übertragen und wirkt sich somit unter Umständen auch auf die Gesamtbewertung eines Artefaktes aus.
\end{quote}

Gestützt wird dieser Gedanke durch Studien von \cite{lindgaard2006attention}, \cite{campbell1996fitting}, und \cite{nickerson1998confirmation}.

Unterstützend seien hier noch zwei weitere Quellen aufgeführt, die die Relevanz weiter unterstreichen:
\cite{tractinsky2006evaluating} zeigen, dass die Ergebnisse der Studie von Lindgaard et al. auch mit anderen Parametern bestand haben und unterstreichen weiterhin die Wichtigkeit von guter Gestaltung für eine gute Nutzererfahrung \cite{tractinsky2000beautiful}.

Neben der Relevanz des Produktes, das in Rahmen der Arbeit entstehen soll ist aber auch das eigentliche Thema der Arbeit zu rechtfertigen: Die Entwicklung von einem Konzept zu einem fertigen Produkt.
Als abschließende Arbeit für den Studiengang Medieninformatik ist dieses ein passendes Thema, da in diesem viele Aspekte aus verschiedenen Modulen des gesamten Studiums vereint werden. Daher bietet das Thema eine gute Verbindung zwischen den verschiedenen Disziplinen innerhalb des Studiums, verbunden mit einer wissenschaftlichen Diskussion verschiedener Vorgehensweisen und Abläufe.

\section{Abgrenzung}
Auch wenn es Teil der Zielsetzung ist, ein marktfähiges Produkt zu erstellen kann hier nicht davon ausgegangen werden, dass das Endergebnis der Arbeit ein Produkt ist, dass als fertig angesehen werden kann. Auch mit Blick auf die spätere Weiterentwicklung durch die Community muss es viel mehr das Ziel sein, eine hochwertige erste Version des Produktes, die als Grundlage für weitere Features dient, also ein \textit{Minimum Viable Product} zu erstellen.

\section{Zielgruppe}
Während der Konzeption im Praxisprojekt wurden Studenten der Technischen Hochschule Köln im Studiengang Medieninformatik als Zielgruppe festgelegt. Es wurde aber bereits im Praxisprojekt deutlich, dass diese Zielgruppe leicht erweiterbar ist. Somit kann jede Person einen Mehrwert aus diesem Tool ziehen, der ein Artefakt erstellen muss, dessen Hauptaugenmerk eigentlich nicht auf der Gestaltung, sondern auf einer bestimmten Funktion liegt. Das kann im Studium an einer mangelnden Bewertung oder in der Wirtschaft an einem mangelndem Budget liegen, jedoch entstehen häufig Situationen, in denen eine grundlegend solide Gestaltung nicht als wichtig erachtet wird, jedoch durchaus Vorteile mit sich bringt.

Für diese Arbeit werden aber zunächst weiterhin die Studenten des Studienganges Medieninformatik als Zielgruppe definiert, um eine Disparität zwischen dem Konzept und der Umsetzung auszuschließen.

\section{Struktur der vorliegenden Arbeit}
Außerhalb dieser Einleitung Teilt sich die Arbeit in drei weitere Kapitel. Im zweiten Kapitel werden zunächst einige Theoretische Grundlagen erläutert und Entscheidungen auf einer taktischen Ebene erläutert. Im dritten Kapitel werden einige Aspekte der Umsetzung der Anwendung erläutert, während im vierten Kapitel einige Fragen bezüglich der Veröffentlichung der Anwendung beantwortet werden.

\textbf{NOTIZ: Hier wird am Ende noch einmal drüber geschaut, wenn die tatsächliche Struktur etwas deutlicher ist.}
