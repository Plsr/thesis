\chapter{Einleitung}
\thispagestyle{fancy}
Diese Arbeit beschäftigt sich mit der Umsetzung der Webanwendung \textit{25knots}, die eine Verbesserung der gestalterischen Qualität von Artefakten im Hochschulkontext sichern soll.
Das Konzept für die Anwendung wurde vorhergehend bereits im Praxisprojekt erarbeitet und beispielhaft in Form eines Proof of Concept umgesetzt. Im Rahmen der Abschlussarbeit soll nun, aufbauend auf den erarbeiteten Konzepten und dem Proof of Concept, ein marktfähiges Produkt erstellt werden. Dieses Produkt soll weiterhin von der Community\footnotemark nicht nur verwendet, sondern auch aktiv weiterentwickelt werden können.\\

Generell sollen in dieser Arbeit alle Aspekte behandelt werden, die auf dem Weg von einem Prototypen zu einem marktfähigen Produkt eine Rolle spielen. Diese umfassen zum Beispiel die Struktur und Gestaltung der Anwendung, technische Entscheidungen die bei der Umsetzung getroffen werden müssen, aber auch Bereiche wie Hosting oder Möglichkeiten zur Weiterentwicklung.
Im Folgenden sollen zunächst einige grundlegende Ziele und das generelle Vorgehen bei der Umsetzung definiert werden.

\footnotetext{Als Teil der \textit{Community} wird hier jede Person gesehen, die ein Interessen an der Anwendung besitzt.}

\section{Motivation}
Wie bereits erwähnt wurde das Konzept für die Anwendung bereits im Praxisprojekt erstellt. Dabei wurde viel Zeit und Energie investiert um sicher zu stellen, dass diese Konzepte auch Umsetzbar sind. So wurden zum Beispiel bereits viele (grobe) Algorithmen entworfen, die in der Anwendung verwendet werden können.
Des weiteren konnte in persönlichen Gesprächen mit verscheidenen Personen innerhalb und auch außerhalb des Hochscholkontextes festgestellt werden dass eine Umsetzung der erarbeiteten Konzepte durchaus eine Relevanz besitzt und von Personen verwendet werden würde.\\

Die Motivation für diese Arbeit ergibt sich daher aus dem Willen, aus diesem erarbeiteten Konzept einen Mehrwert schaffen zu wollen, der über das verdienen von Credit Points hinaus geht. Die generelle Motivation für eine Anwendung wie \textit{25knots} lässt sich Kapitel \textit{\ref{sec:relevance}: \nameref{sec:relevance}} entnehmen

\section{Zielsetzung}
Ziel der Abschlussarbeit soll es sein, ein verwendbares Produkt zu erstellen. \textit{Verwendbar} lässt sich hierbei in drei Unterzielen definieren:

\begin{enumerate}
  \item Die Anwendung muss einen hohe Qualität aufweisen, um eine befriedigende Nutzererfahrung zu bieten
  \item Die Anwenung muss öffentlich zugänglich sein
  \item Potentielle Nutzer müssen von der Existenz der Anwendung wissen
\end{enumerate}

Das Hauptaugenmerk der Arbeit wird dabei auf der sicherung der Qualität das Produktes liegen. Hier spielen sowohl die Programmierung der Anwendung, als auch das Design (Visuell und Strukturell) eine Schlüsselrolle. \\
Um die Anwendung öffentlich zugänglich zu machen muss eine Entscheidung über das Hosting getroffen werden. Dabei soll der Dienst \textit{Heroku} verwendet werden. Dieser bietet die Möglichkeit, ohne übermäßige Konfiguration Inhalte zu hosten und lässt sich nach Bedarf erweitern, um beispielsweise höhere Nutzerzahlen abbilden zu können. \\
Damit potentielle Nutzer Kenntnis von der Anwendung erhalten, muss diese beworben werden. Mit Blick auf die Zielgruppe bietet sich zunächst eine Bewerbung direkt an der Hochschule, beispielweise durch die Teilnahme am \textit{Medieninfromatik Showcase} an. Aber auch eine Bewerbung im weiteren Rahmen, beispielsweise durch einen Talk beim \textit{Webmontag Köln} ist denkbar.

Zuletzt soll auch eine mögliche spätere Weiterentwicklung der Anwendung bedacht werden, denn im Rahmen der Abschlussarbeit können unmöglich alle für die Gestaltung wichtigen Themen abgedeck werden. Der Quellcode der Anwendung soll auf der Plattform \textit{Github} öffentlich zur Verfügung stehen, so dass auch Dritte an der Weiterentwicklung beteiligt sein können. Mit Blick auf die Beiteiligung dritter muss es auch Ziel der Arbeit sein, den Quellcode gut zu strukturieren und dokumentieren, um einen Einsteig für neue Personen möglichst einfach zu machen.

\section{Relevanz des Themas}
\label{sec:relevance}
BDie Relevanz der Anwendung an sich wurde bereits im Praxisprojekt erläutert, daher soll hier nur eine Kurzfassung der Erläuterung folgen:\\
Lindgaard et al. \cite{lindgaard2006attention} haben gezeigt, dass Menschen sich in nur 50ms ein Urteil über die Gestaltung einer Website bilden. Der erste Eindruck spielt bei Menschen auch bei der späteren Bewertung einer Sache noch eine wichtige Rolle \cite{campbell1996fitting}. Von diesem ersten Eindruck lassen sich Menschen nur schwer wieder abbringen \cite{nickerson1998confirmation}.
Daher ist es wichtig, dass auch Artefakte in Modulen, in denen die Gestaltung gegebenenfalls nicht bewertet wird, ein solides Design aufweisen, um den ersten Eindruck so positiv wie möglich zu gestalten.

Unterstützend seien hier noch zwei weitere Quellen aufgeführt, die die Relevanz weiter unterstreichen:\\
\cite{tractinsky2006evaluating} zeigen, dass die Ergebnisse der Studie von Lindgaard et al. auch mit anderen Parametern bestand haben und unterstreichen weiterhin die Wichtigkeit von guter Gestaltung für eine gute Nutzererfahrung \cite{tractinsky2000beautiful}.\\
Auch aus vielen persönlichen Gespräche konnte ich entnehmen, dass die Umsetzung der erarbeiteten Konzepte als Produkt, auch für verschiedene Personengruppen, durchaus wünschenswert ist.

Neben der Relevanz des Produktes, das in Rahmen der Arbeit entstehen soll ist aber auch das eigentliche Thema der Arbeit zu rechtfertigen: Die Entwicklung von einem Konzept zu einem fertigen Produkt.
Als abschließende Arbeit für den Studiengang Medieninformatik ist dies ein passendes Thema, da in diesem viele Aspekte des gesamten Studiums vereint werden. In den weiter oben genannten Bereichen kommen beispielsweise Inhalte aus den Modulen GdvK, WBA 1, ST,  PM und EIS vor. Daher bietet das Thema eine gute Verbindung zwischen den verschiedenen Disziplinen innerhalb des Studiums, verbunden mit einer wissenschaftlichen Diskussion verschiedener Vorgehensweisen und Abläufe.

\section{Abgrenzung}
Bei der Textgestaltung und automatischen Änderung von Abbildungsnummern, Querverweisen,
Seitenzahlen, Gliederungen, Literaturhinweisen etc. bietet sich der Rückgriff
auf moderne Textverarbeitungsprogramme an. Nutzen Sie diese zur besseren Lesbarkeit
und Strukturierung des Textes, aber vermeiden Sie überflüssige Spielereien. Da
besonders bei Textdokumenten mit eingebundenen Objekten wie Bildern, Formeln

\section{Zielgruppe}
Bei der Textgestaltung und automatischen Änderung von Abbildungsnummern, Querverweisen,
Seitenzahlen, Gliederungen, Literaturhinweisen etc. bietet sich der Rückgriff
auf moderne Textverarbeitungsprogramme an. Nutzen Sie diese zur besseren Lesbarkeit
und Strukturierung des Textes, aber vermeiden Sie überflüssige Spielereien. Da
besonders bei Textdokumenten mit eingebundenen Objekten wie Bildern, Formeln

\section{Struktur der vorliegenden Arbeit}
asdf
asd
asd

asd
asd
asd
