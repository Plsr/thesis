\chapter{Einleitung}
\thispagestyle{fancy}
ie Blätter im Textteil sind fortlaufend zu nummerieren. Die Seitenzahl steht auf dem
Blattrand unten rechts und ist auf dem Deckblatt zu unterdrücken. Die Seitennummerierung
endet mit dem Anhang.

\section{Motivation}
Für den reinen Textteil sollten übliche Schrifttypen (Times New Roman 12 pt. oder Arial
11 pt.) verwendet werden und die Arbeit auf DIN A4-Papier, einseitig beschriftet, erstellt
werden. Der Zeilenabstand sollte 1,5-fach erfolgen. Für Abschlussarbeiten sind ungefähre
Mindest-Seitenränder von links 3 cm (aufgrund der Bindung), rechts ca. 2 cm,
unten 2 - 3 cm und oben 1,5-2,5 cm bis zur Seitenzahl sowie weitere 1-1,5 cm von der
Seitenzahl zum Textbeginn einzuhalten.
Die Blätter im Textteil sind fortlaufend zu nummerieren. Die Seitenzahl steht auf dem
Blattrand unten rechts und ist auf dem Deckblatt zu unterdrücken. Die Seitennummerierung
endet mit dem Anhang.
Während Briefe immer linksbündig geschrieben werden, ist die Form der wissenschaftlichen
Berichte immer der Blocksatz. In beiden Fällen lohnt sich die automatische Silbentrennung.
Die Blätter im Textteil sind fortlaufend zu nummerieren. Die Seitenzahl steht auf dem
Blattrand unten rechts und ist auf dem Deckblatt zu unterdrücken. Die Seitennummerierung
endet mit dem Anhang.
Während Briefe immer linksbündig geschrieben werden, ist die Form der wissenschaftlichen
Berichte immer der Blocksatz. In beiden Fällen lohnt sich die automatische Silbentrennung.
Die Blätter im Textteil sind fortlaufend zu nummerieren. Die Seitenzahl steht auf dem
Blattrand unten rechts und ist auf dem Deckblatt zu unterdrücken. Die Seitennummerierung
endet mit dem Anhang.
Während Briefe immer linksbündig geschrieben werden, ist die Form der wissenschaftlichen
Berichte immer der Blocksatz. In beiden Fällen lohnt sich die automatische Silbentrennung.

Bei der Textgestaltung und automatischen Änderung von Abbildungsnummern, Querverweisen,
Seitenzahlen, Gliederungen, Literaturhinweisen etc. bietet sich der Rückgriff
auf moderne Textverarbeitungsprogramme an. Nutzen Sie diese zur besseren Lesbarkeit
und Strukturierung des Textes, aber vermeiden Sie überflüssige Spielereien. Da
besonders bei Textdokumenten mit eingebundenen Objekten wie Bildern, Formeln

\section{Zielsetzung}
Bei der Textgestaltung und automatischen Änderung von Abbildungsnummern, Querverweisen,
Seitenzahlen, Gliederungen, Literaturhinweisen etc. bietet sich der Rückgriff
auf moderne Textverarbeitungsprogramme an. Nutzen Sie diese zur besseren Lesbarkeit
und Strukturierung des Textes, aber vermeiden Sie überflüssige Spielereien. Da
besonders bei Textdokumenten mit eingebundenen Objekten wie Bildern, Formeln

\section{Relevanz des Themas}
Bei der Textgestaltung und automatischen Änderung von Abbildungsnummern, Querverweisen,
Seitenzahlen, Gliederungen, Literaturhinweisen etc. bietet sich der Rückgriff
auf moderne Textverarbeitungsprogramme an. Nutzen Sie diese zur besseren Lesbarkeit
und Strukturierung des Textes, aber vermeiden Sie überflüssige Spielereien. Da
besonders bei Textdokumenten mit eingebundenen Objekten wie Bildern, Formeln

\section{Abgrenzung}
Bei der Textgestaltung und automatischen Änderung von Abbildungsnummern, Querverweisen,
Seitenzahlen, Gliederungen, Literaturhinweisen etc. bietet sich der Rückgriff
auf moderne Textverarbeitungsprogramme an. Nutzen Sie diese zur besseren Lesbarkeit
und Strukturierung des Textes, aber vermeiden Sie überflüssige Spielereien. Da
besonders bei Textdokumenten mit eingebundenen Objekten wie Bildern, Formeln
