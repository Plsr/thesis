\chapter{Veröffentlichung der Anwendung}
\thispagestyle{fancy}

Der letzte Schritt, der für die Entwicklung einer marktfähigen Webanwendung von Bedeutung ist, ist deren Veröffentlichung. Dabei müssen in diesem Fall zwei verschiedene Aspekte angesprochen werden: Zum Einen muss die Anwendung für den Endnutzer verfügbar sein, zum Anderen ist es aber auch Ziel dieser Arbeit, eine spätere Weiterentwicklung der Anwendung durch die Community zu ermöglichen. Daher muss auch der Quellcode der Anwendung zugänglich und Möglichkeiten zur Mitarbeit definiert sein.

\section{Ausliefern der Anwendung}
\label{chap:hosting}
Ein Verfügbar machen von Ressourcen und Anwendungen im World Wide Web suggeriert in der Regel die Verwendung eines Servers zum ausliefern dieser Ressourcen. Da es sich bei dieser Anwendung um eine \textit{Single Page Application} handelt, sind die technischen Ansprüche an einen solchen Server äußerst gering. Dieser muss lediglich in der Lage sein, zwei statische Dateien auszuliefern.
Hier sollen zunächst drei Lösungsmöglichkeiten näher betrachtet werden:

\begin{itemize}
  \item das manuelle verwalten eines Servers
  \item die Verwendung von \textit{Github Pages}
  \item die Verwendung des PaaS \textit{heroku}
\end{itemize}

Ein manuell verwalteter Server bietet von allen Möglichkeiten die flachste Lernkurve. Es müssen keine neuen Konzepte erlernt werden und aufgrund des niedrigen technischen Anspruches muss keine komplizierte Konfiguration des Servers erfolgen.\\
Von Nachteil ist hierbei jedoch die niedrige Flexibilität. Jede neue Version der Anwendung muss manuell auf dem Server abgelegt werden oder ein Automatismus muss implementiert werden.

Da der Quellcode der Anwendung auf Github veröffentlich wurde (siehe dazu Kapitel ASD), ist die verwendung von \textit{Github Pages} eine naheliegende Option. Die Erweiterung der Plattform erlaubt es, den Quellcode eines bestimmten \textit{branches} innerhalb eines \textit{Repository} statisch verfügbar zu machen.\\
Die veröffentlichung einer neuen Version der Anwendung ist dabei sehr einfach, außerdem muss keine weitere Plattform in den Ablauf integriert werden. Während der Tests mit \textit{Github Pages} kam es jedoch zu Konflikten mit einer der verwendeten Bibliotheken. Auch wenn diese Konflikte lösbar waren, lassen diese doch auf mögliche Probleme in der Zukunft schließen.

\textit{Heroku} erlaubt die Veröffentlichung von Komplexen Anwendung mit wenig Konfiguration. Der Service ist dabei auf das asuliefern von dynamischen Backends spezialisiert und eigentlich nicht für das Ausliefern von statischen Dateien gedacht. \textit{Heroku} bietet jedoch von allen genannten Lösungen die einfachsten Möglichkeiten, Prozesse zu automatisieren und zu skalieren.

Auch wenn alle hier angesprochenen Lösungen von einem technischen Standpunkt gesehen für die Auslieferung der Anwendung in Frage kommen würden, wird vor dem Hintergrund der möglichen Weiterentwicklung der Anwendung in Zukunft \textit{Heroku} verwendet.

Für eine leichte Zugänglichkeit ist außerdem eine prägnante URL, unter der die Anwendung aufgerufen werden kann, von Vorteil. Aus diesem Grund wurde die Domain \url{25knots.de} registriert, unter der die Anwendung abgerufen werden kann.

\section{Weiterentwicklung}
Nachfolgend soll außerdem auf eine mögliche Weiterentwicklung der Anwendung nach dem Abschluss dieser Arbeit eingegangen werden. In diesem Zusammenhang stellen sich vor allem drei Fragen:
\begin{enumerate}
  \item Wie können Personen aus der Community dazu gebracht werden, an einer Weiterentwicklung mitzuarbeiten?
  \item Wie kann eine durchgängig hohe Qualität des Quellcodes garantiert werden, auch wenn viele verschiedene Menschen daran arbeiten?
  \item Wie kann das Mitwirken für Interessenten möglichst einfach gestaltet werden?
\end{enumerate}

Eine konkrete Beantwortung der ersten Frage gestaltet sich recht schwierig. Es lässt sich jedoch ein logischer Zusammenhang zwischen der Anzahl der Nutzer und der Anzahl der Mitwirkenden eines Projektes feststellen. Am Beispiel der Plattform Github können \textit{stars} als relativ sichere Mindestzahl der Nutzer gesehen werden. Laut \cite{borges2015popularity} besteht ein Zusammenhang zwischen \textit{stars} und der anzahl der Mitwirkenden an einem Projekt:

\begin{quote}
  In git-based systems, forks are used to either propose changes to an application or as
a starting point for a new project. In both cases, the number of forks can be seen as a proxy
for the importance of a project in GitHub. [...] Two facts can be observed in this figure. First, there is a strong positive
correlation between stars and forks (Spearman rank correlation coefficient = 0.55). Second,
only a few systems have more forks than stars.
\end{quote}

Somit ist eine Steigerung der Mitwirkenden also automatisch mit einer Steigerung der Nutzer der Anwendung verbunden. Der Inhalt diese Abschnittes soll also vorrangig mit der Beantwortung der zweiten und dritten Frage beschäftigen.

\subsection{Sicherung der Code-Qualität}
Nach der Veröffentlichung der Anwendung ist es beabsichtigt, dass weitere Leute den Quellcode der Anwendung erweitern und bearbeiten. Hier kann nicht von einem festen Team ausgegangen werden, so ist es durchaus auch möglich, dass einzelne Personen nur ein einziges Mal und an einem kleinen Teil er Anwendung arbeiten. Für die Anwendung ist es von Vorteil einen durchgängigen Code-Stil zu verwenden, um die Lesbarkeit durch die verschiedenen Bereiche der Anwendung zu verbessern. Personen, die allerdings nur für einen kurzen Moment Kontakt mit der Anwendung haben, kann  jedoch nicht abverlangt werden, dass diese sich vorher eingehend mit den verschiedenen Code-Styles der Anwendung befasst haben. Uns diesem Grund wurde die Bibliothek eslint eingebunden, zur Unterstützung der Einhaltung von Coding-Styles innerhalb einer Anwendung dient.
In der Datei .eslintrc wird dabei festgelegt, welche Conventions befolgt werden sollen. Durch das einbinden des Bibliothek im Webpack-Build-Prozess können dann Warnung und error-messages ausgegeben werden, wenn gegen die Guidelines verstoßen wurde. Für einige Verstöße (wie zum Beispiel Fehler bei der Einrückung) lässt sich auch eine automatische  Fehlerbehebung einstellen.

Auch wenn die Fehlermeldungen Webpack daran hindern, die Anwendung zu kompilieren, ist es dennoch möglich, Code auf GitHub zu pushen, der gegen die Conventions verstößt. Um hier eine weiter Sicherung einzubauen, wurden die beiden Branchen master und development, die jeweils in Verbindung mit der Staging- und Production-Umgebung stehen, geschützt. In diese Branches kann also nicht mehr direkt committed werden, sondern nur noch über Pull Requests.

Für die Überprüfung von neuen Pull Request wurde der Service Travis eingebunden. Dieser service kann bei jedem neuen Pull Request bestimmte Programme ablaufen lassen und nur wenn diese keine Fehler ausgeben, kenn der Pull Request gemerged werden (s. Abb ZXC). Dies bietet sich zum Beispiel für das Ausführen von Tests an, aber auch für das testen der Code-Styles mit eslint.

\subsection{Dokumentation}
Grober Aufbau für diese Sektion:

\begin{enumerate}
  \item Was zeichnet eine gute Dokumentation aus?
  \item Dokumentation innerhalb des Codes
  \item Dokumentation außerhalb des Codes
\end{enumerate}

\section{Vermarktung}
Bei der Textgestaltung und automatischen Änderung von Abbildungsnummern, Querverweisen,
Seitenzahlen, Gliederungen, Literaturhinweisen etc. bietet sich der Rückgriff
auf moderne Textverarbeitungsprogramme an. Nutzen Sie diese zur besseren Lesbarkeit
und Strukturierung des Textes, aber vermeiden Sie überflüssige Spielereien. Da
besonders bei Textdokumenten mit eingebundenen Objekten wie Bildern, Formeln
