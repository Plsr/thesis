\chapter{Veröffentlichung der Anwendung}
\thispagestyle{fancy}

Bei der Textgestaltung und automatischen Änderung von Abbildungsnummern, Querverweisen,
Seitenzahlen, Gliederungen, Literaturhinweisen etc. bietet sich der Rückgriff
auf moderne Textverarbeitungsprogramme an. Nutzen Sie diese zur besseren Lesbarkeit
und Strukturierung des Textes, aber vermeiden Sie überflüssige Spielereien. Da
besonders bei Textdokumenten mit eingebundenen Objekten wie Bildern, Formeln

\section{Hosting}
\label{chap:hosting}
Um den Zugang zur Anwendung für die Community möglichst einfach zu gestalten, bietet es sich an, diese zu hosten (einer andere, weniger geeignete Möglichkeit, wäre beispielsweise nur die Veröffentlichung des Quellcodes und das lokale hosten durch den Nutzer selbst).
Da es sich bei der Anwendung um eine rein Clientseitige handelt, stehen eine Vielzahl von Möglichkeiten für das hosting zu Verfügung, da der Server lediglich statische html- und JavaScript-Dateien ausliefern muss.

In die nähere Auswahl kamen in diesem Projekt Github Pages und Heroku. Ein hosting auf einem privaten Server wurde schnell ausgeschlossen, um die mögliche zusätzliche Arbeit möglichst gering zu halten.
Da der Quellcode der Anwendung bereits auf Github veröffentlich wurde, scheint das Hosting über Github Pages zunächst die naheliegendste und einfachste Lösung. Ein Deployment auf Github Pages erfolgt nach entsprechenden Einstellungen im Repository einfach durch einen push auf den gh-pages branch. Die Anwenudung ist danach über die Domain username.github.io/rpository erreichbar. Größter Vorteil ist dabei, dass kein externer Service benötigt wird. Es treten aber auch einige Limitierungen auf. Beim testen der Hosting-Möglichkeiten wurde deutlich, dass das Modul React-Router den zusätzlichen Path nicht ohne weiteres unterstützt. Zwar bieten sich hier relative einfach Lösungen wie das verwenden von hashes anstatt von Pfaden im React-Router an, jedoch wurde mit blick auf die bereits bei simplen dingen auftretenden Probleme diese Möglichkeit zunächst als weniger geeignet eingestuft.

Für das Hosting wurde sich für den Service Heroku entschieden. Heroku ist ein Saas, das ein Deployment und hosting ohne Setup und Konfiguration erlaubt. Ein Deployment auf Heroku läuft dabei über einen remote branch im git repository, kann also mit dem Befehl \verb|git push heroku master| gestartet werden. Heroku erkennt die verwendete Sprache automatisch und stellt alles nötige automatisch ein.
Jedoch unterstützt Heroku lediglich Serverseitige Progrmmiersprachen und beitet keinen Support für das Hosting von statischen files. Um diese Anwendung zu hosten, muss also ein Server geschreiben werden. Da dieser nur statische Files hosten muss, ist dieser sehr simple in Node.js und dem Framework express geschrieben:

\begin{lstlisting}
  const express = require('express')
  const app = express()

  app.use(express.static(__dirname + '/dist'))
  app.listen(process.env.PORT || 8080)
\end{lstlisting}

Der Server hostet dabei den ordner \verb|dist|, in dem sich die gebuildeten Files befinden (unter anderem auch die Datei \verb|index.html|, auf die ein Browser automatisch zurück greift). Mit der Datei \verb|Procfile| wird Heroku dann mitgeteilt, welche Befehle beim Deployment der Anwendung ausgeführt werden soll. Im Falles dieser Anwendung ist das die Datei \verb|server.js|. Das \verb|Procfile| besteht alsolediglich aus der Zeile

\begin{lstlisting}
  web: node server.js
\end{lstlisting}

Da sowohl Github Pages, als auch Heroku von sich aus eine recht kryptische URL verwenden, wurde weiterhin die Domain \verb|25knots.de| gekauft, um auch hier den Einstieg für potentielle Nutzer so einfach wie möglich zu gestalten.


\section{Weiterentwicklung}
Nachfolgend soll außerdem auf eine mögliche Weiterentwicklung der Anwendung nach dem Abschluss dieser Arbeit eingegangen werden. In diesem Zusammenhang stellen sich vor allem drei Fragen:
\begin{enumerate}
  \item Wie können Personen aus der Community dazu gebracht werden, an einer Weiterentwicklung mitzuarbeiten?
  \item Wie kann eine durchgängig hohe Qualität des Quellcodes garantiert werden, auch wenn viele verschiedene Menschen daran arbeiten?
  \item Wie kann das Mitwirken für Interessenten möglichst einfach gestaltet werden?
\end{enumerate}

Eine konkrete Beantwortung der ersten Frage gestaltet sich recht schwierig. Es lässt sich jedoch ein logischer Zusammenhang zwischen der Anzahl der Nutzer und der Anzahl der Mitwirkenden eines Projektes feststellen. Am Beispiel der Plattform Github können \textit{stars} als relativ sichere Mindestzahl der Nutzer gesehen werden. Laut \cite{borges2015popularity} besteht ein Zusammenhang zwischen \textit{stars} und der anzahl der Mitwirkenden an einem Projekt:

\begin{quote}
  In git-based systems, forks are used to either propose changes to an application or as
a starting point for a new project. In both cases, the number of forks can be seen as a proxy
for the importance of a project in GitHub. [...] Two facts can be observed in this figure. First, there is a strong positive
correlation between stars and forks (Spearman rank correlation coefficient = 0.55). Second,
only a few systems have more forks than stars.
\end{quote}

Somit ist eine Steigerung der Mitwirkenden also automatisch mit einer Steigerung der Nutzer der Anwendung verbunden. Der Inhalt diese Abschnittes soll also vorrangig mit der Beantwortung der zweiten und dritten Frage beschäftigen.

\subsection{Sicherung der Code-Qualität}
Nach der Veröffentlichung der Anwendung ist es beabsichtigt, dass weitere Leute den Quellcode der Anwendung erweitern und bearbeiten. Hier kann nicht von einem festen Team ausgegangen werden, so ist es durchaus auch möglich, dass einzelne Personen nur ein einziges Mal und an einem kleinen Teil er Anwendung arbeiten. Für die Anwendung ist es von Vorteil einen durchgängigen Code-Stil zu verwenden, um die Lesbarkeit durch die verschiedenen Bereiche der Anwendung zu verbessern. Personen, die allerdings nur für einen kurzen Moment Kontakt mit der Anwendung haben, kann  jedoch nicht abverlangt werden, dass diese sich vorher eingehend mit den verschiedenen Code-Styles der Anwendung befasst haben. Uns diesem Grund wurde die Bibliothek eslint eingebunden, zur Unterstützung der Einhaltung von Coding-Styles innerhalb einer Anwendung dient.
In der Datei .eslintrc wird dabei festgelegt, welche Conventions befolgt werden sollen. Durch das einbinden des Bibliothek im Webpack-Build-Prozess können dann Warnung und error-messages ausgegeben werden, wenn gegen die Guidelines verstoßen wurde. Für einige Verstöße (wie zum Beispiel Fehler bei der Einrückung) lässt sich auch eine automatische  Fehlerbehebung einstellen.

Auch wenn die Fehlermeldungen Webpack daran hindern, die Anwendung zu kompilieren, ist es dennoch möglich, Code auf GitHub zu pushen, der gegen die Conventions verstößt. Um hier eine weiter Sicherung einzubauen, wurden die beiden Branchen master und development, die jeweils in Verbindung mit der Staging- und Production-Umgebung stehen, geschützt. In diese Branches kann also nicht mehr direkt committed werden, sondern nur noch über Pull Requests.

Für die Überprüfung von neuen Pull Request wurde der Service Travis eingebunden. Dieser service kann bei jedem neuen Pull Request bestimmte Programme ablaufen lassen und nur wenn diese keine Fehler ausgeben, kenn der Pull Request gemerged werden (s. Abb ZXC). Dies bietet sich zum Beispiel für das Ausführen von Tests an, aber auch für das testen der Code-Styles mit eslint.

\subsection{Dokumentation}
Grober Aufbau für diese Sektion:

\begin{enumerate}
  \item Was zeichnet eine gute Dokumentation aus?
  \item Dokumentation innerhalb des Codes
  \item Dokumentation außerhalb des Codes
\end{enumerate}


\subsection{Contribution Guidelines}
zahlen, Gliederungen, Literaturhinweisen etc. bietet sich der Rückgriff
auf moderne Textverarbeitungsprogramme an. Nutzen Sie diese zur besseren Lesbarkeit
und Strukturierung des Textes, aber vermeiden Sie überflüssige Spielereien. Da
besonders bei Textdokumenten mit eingebundenen Objekten wie Bildern, Formeln

\subsection{Tests}
zahlen, Gliederungen, Literaturhinweisen etc. bietet sich der Rückgriff
auf moderne Textverarbeitungsprogramme an. Nutzen Sie diese zur besseren Lesbarkeit
und Strukturierung des Textes, aber vermeiden Sie überflüssige Spielereien. Da
besonders bei Textdokumenten mit eingebundenen Objekten wie Bildern, Formeln

\section{Vermarktung}
Bei der Textgestaltung und automatischen Änderung von Abbildungsnummern, Querverweisen,
Seitenzahlen, Gliederungen, Literaturhinweisen etc. bietet sich der Rückgriff
auf moderne Textverarbeitungsprogramme an. Nutzen Sie diese zur besseren Lesbarkeit
und Strukturierung des Textes, aber vermeiden Sie überflüssige Spielereien. Da
besonders bei Textdokumenten mit eingebundenen Objekten wie Bildern, Formeln
